\documentclass{ltjarticle}
\title{太陽・恒星物理学を始める上で必ず履修しておくべき物理学+$\alpha$}
\author{名古屋大学宇宙地球環境研究所 堀田英之}
\date{\today}

\usepackage[margin=20truemm]{geometry}
\usepackage[pdfencoding=auto,colorlinks=true]{hyperref}
\usepackage{url}
\usepackage{bm}
\usepackage{amsmath,amssymb}
\newcommand{\red}[1]{{\textcolor{red}{#1}}}
%\newenvironment{en}{\begin{enumerate}\setcounter{enumi}{\value{mymemory}}}{\setcounter{mymemory}{\value{enumi}}\end{enumerate}}
%\newcommand{\bm}[1]{{\mbox{\boldmath $#1$}}}
%\usepackage[dvipdfmx]{graphicx}
\usepackage{mathrsfs}
\usepackage[luatex]{graphicx,xcolor}
\usepackage{enumitem}
\usepackage{wrapfig}
\numberwithin{equation}{section}
\usepackage{fancyhdr,fancybox}
\pagestyle{fancy}

\begin{document}
\maketitle
\section{はじめに}
この文章では、堀田のもとで太陽・恒星物理学の研究を始める上で「最低限」必要な物理学の知識についてまとめます。少なくともこれだけあれば、何とか研究に必要な勉強を始められるというものです。会話する上で、これは知っていることが前提とされると言い換えても構いません。\par
これ以上やってはいけないという意味ではまったくありませんし、むしろ余裕があるならば、やるべきです。この文章では、たとえば統計力学や量子力学には触れません。太陽・恒星物理学の磁気流体力学のみを研究対象にするならば、これらは必ずしも必要ではありません。しかし、これらは一般の物理学科を卒業した学生ならば履修しており、その考え方は非常に有用です。例を挙げるならば、量子力学で登場するBorn近似は、日震学にも使われより適切な太陽内部の波動伝播をモデル化しています。また、将来研究者、とくに大学教員になりたいみなさんは広い物理学を理解しておくことは、採用の際に重要となります。さらには、そもそもこれらの学問を学ぶことは非常に楽しいです。
\par
しかし、近年では、私のもとに訪れる学生が必ずしも物理学科出身でないことが増えました。それは1つ興味深いことであるのですが、彼/彼女らに物理学科で4年で学ぶ事柄を大学院に入ってからすべて追ってもらうのは得策でないことは明らかです。そこで、「最低限」必要な物理学をまとめておくことに思い至りました。
\par
ひとまずということで、

\subsection{この文章の読み方}
この文章では、答えを示さないままいくつかの知っておいてほしい物理学的な事柄を示します。教科書を見ながら、それを自分で深くまで理解し、咀嚼してください。\par
物理学では、多くの数式が登場しますが、それを物理的に説明することが重要です。漫然と式変形するのではなく、それぞれの段階で数式がどのような現実の状況を表しているかを説明できるように意識してください。\par
また、式を導く際は、「次元」に細心の注意を払いましょう。次元とは単位のことです。たとえば重力下の運動方程式であれば
\begin{align}
    m\frac{d\bm{v}}{dt} = m \bm{g}
\end{align}
です。$m$は$\mathrm{g}$、$\bm{v}$は$\mathrm{cm~s^{-1}}$、$t$は$s$の単位を持つので、左辺は$\mathrm{g~cm~s^{-2}}$の次元を持っています。一方、右辺は$g$が$\mathrm{cm~s^{-2}}$の次元を持っているので、同様に$\mathrm{g~cm~s^{-2}}$の次元を持っています。当然、左辺と右辺の次元は同じになるのです。また、次元の違うもの同士を足し弾きすることはできません。これが守られていないと、物理をさっぱり理解していない証明となってしまいます。よく注意して式変形を行いましょう。
\subsection{記法}
この文章では、ベクトルの演算は$\nabla$、$\nabla\cdot$、$\nabla\times$を用いて行われ$\mathrm{grad}$、$\mathrm{div}$、$\mathrm{rot}$は使われません。学生の皆さんにもそのようにすることを強く推奨します。前者は、微分演算をあたかもベクトルのように扱っており、その正体を掴みやすいです。

\section{物理数学}
\subsection{微積分}
\begin{itemize}
    \item 微分に数学的な定義は?
    \item 常微分と偏微分の違いは?
\end{itemize}
\subsection{ベクトル解析}
\subsection{フーリエ変換}
\section{力学}
力学には、多くの魅力的な問題があるが、必ずしもそれらが解ける必要はない(大学院試験では、よく出されますが。。)。それよりも、力学で定義されているいくつかの概念をしっかり理解して、その後に進むことが非常に大切です。
\subsection{座標}
2次元極座標での加速度の動径成分$a_r$、$a_\theta$はそれぞれ
\begin{align}
    a_r &= \ddot{r} - r\dot{\theta}^2 \\
    a_\theta &= r\ddot{\theta} + 2\dot{r}\dot{\theta}
\end{align}
となることを証明せよ
\subsection{運動方程式}
\subsubsection{保存力}
力$\bm{F}$が保存力であるとき
\begin{align}
    \bm{F} = - \nabla U
\end{align}
とポテンシャル$U$を用いて表せることを証明せよ。
\subsection{力学的エネルギー保存}
質点にかかる力が保存力だけの場合に、力学的エネルギーが保存することを証明せよ。

\section{電磁気学}
\section{熱力学}
\end{document}